\documentclass{article}
\usepackage{amsmath, amsthm, amssymb, amsfonts, mathtools, enumitem, stmaryrd,physics, cancel, tikz-cd, graphicx, float, booktabs, mathrsfs}
\usetikzlibrary{arrows}
\usepackage{geometry}
    \geometry{
        a4paper,
        left = 20mm,
        top = 20mm,
        right = 20mm,
        bottom = 30mm
    }
\setlength{\parindent}{0pt}
\setlength{\parskip}{\baselineskip}%

\theoremstyle{definition}
\newtheorem{problem}{Problem}
\newtheorem{solution}{Solution}
\newtheorem*{example}{Example}
\newtheorem*{exercise}{Exercise}
\newtheorem*{definition}{Definition}
\newtheorem{theorem}{Theorem}[section]
\newtheorem*{theorem*}{Theorem}
\newtheorem{proposition}[theorem]{Proposition}
\newtheorem*{proposition*}{Proposition}
\newtheorem{lemma}[theorem]{Lemma}
\newtheorem*{lemma*}{Lemma}
\newtheorem{corollary}[theorem]{Corollary}
\newtheorem*{corollary*}{Corollary}
\newtheorem{remark}[theorem]{Remark}
\newtheorem*{remark*}{Remark}
\newtheorem{notation}[theorem]{Notation}
\newtheorem*{notation*}{Notation}
\newtheorem*{example*}{Example}


\title{M503 Non-Commutative Algebra}
\author{Taught by Matthias Strauch \\ Notes written by Thanic Nur Samin}
\date{}

\begin{document}
    \maketitle

    \tableofcontents

    Office Hours: Wednesdays 4:30, RH316.

    \section*{Tuesday, 1/13/2026}
    
    \section{Introduction}

    We are interested in non-commutative algebras of finite dimension over a field.

    Reference: Lorenz Algebra Vol. 2, chapters 28 and following.

    General convention:

    \begin{enumerate}[label=\roman*)]
        \item Rings will always assumed to be associative and unital.
        \item Modules are always left modules [unless stated otherwise]. Even when philosophically it's better to consider them as right modules we will turn them into the opposite ring.
    \end{enumerate} 

    \(R =\) commutative unital ring.

    \(A =\) unital \(R\)-algebra, not necessarily commutative.

    Recall: this means that \(A\) is an \(R\)-module, and one has \(\forall x\in R, \forall a,b\in A, x \cdot (ab) = (xa) \cdot b = a \cdot (xb)\).

    \begin{remark}
        \begin{enumerate}[label=\arabic*)]
            \item \(\varphi : R \to A, x \mapsto x \cdot 1_A\) is a ring homomorphism with image in \(Z(A) = \{ a\in A \mid \forall b\in A, ab = ba \}\). 
            
            \[
                (x \cdot 1) \cdot a = x \cdot (1 \cdot a) = x \cdot a = x \cdot (a \cdot a) = a \cdot (x \cdot 1_A)
            \]

            \item Conversely, if \(\psi : R \to B\) [where \(B\) is any unital ring] is an unital ring homomorphism with \(\operatorname{im} (\psi) \subset Z(B)\), then the multiplication \(x \cdot b \coloneqq \psi(x) b\) gives \(B\) the structure of an \(R\)-module and \(B\) becomes an \(R\)-algebra.
            \item The map \(\psi\) in (1) need not be injective. \(\mathbb{Z} = R, \mathbb{Z} / n\mathbb{Z} = A\) is allowed.
            \item Subalgebras of an algebra contain the unit element by convention.
            \item By an \textit{ideal} in \(A\) we always mean a \(2\)-sided ideal. We have left ideals and right ideals defined in the usual way.
            \item A \textit{division algebra} is an \(R\)-algebra \(A\) such that \(\forall a \in A \setminus \{ 0_A \} \exists b \in A\) such that \(ab = ba = 1_A\). It is often also called a skew field. The center of a division algebra is a field.
            
            Modules over a division algebra will also be called vector spaces.

            \item \(A\text{-Mod}\) denotes the category of (left) \(A\)-modules.
        \end{enumerate} 
    \end{remark}

    \begin{proposition}
        Let \(A\) be an \(R\)-algebra and \(M,N \in A\text{-Mod}\). Set \(H = \operatorname{Hom}_A(M,N)\). Then \(H\) is an \(R\)-module with the following action: \(R \times H \to  H, (x,f) \mapsto [m \mapsto x \cdot f(m)]\). This gives \(H\) the structure of an \(R\)-module. 
        
        We need \(R\) to be commutative, because otherwise \(am \mapsto x f(am) = xa f(m)\) which is not necessarily \(ax f(m)\). 
        
        If \(N = M\) then \(H = \operatorname{End}_A(M)\) is in fact an \(R\)-algebra w.r.t.\ the same module structure and composition of maps as multiplication.
    \end{proposition}

    \begin{proof}
        HW1
    \end{proof}

    \begin{remark}
        For \(M \in A\text{-Mod}\) we can regard it also as a module over \(\operatorname{End}_R(M)\).
    \end{remark}

    \begin{notation}
        \begin{enumerate}[label=\arabic*)]
            \item \(A_\ell = A\) considered as an \(A\)-module by left multiplication \(A \times A_\ell \to A_\ell, (a,b) \mapsto ab\).
            \item \(A^o =\) opposite algebra of \(A = A\), but with multiplication defined by \(A \underset{A^o}{\cdot} b \coloneqq b \underset{A}{\cdot} a\). \(A^o\) is still an \(R\)-algebra. We set \(A_r = (A^o)_\ell\). This is just \(A\) but with \(A^o\)-module structure.
            
            \[
                A^o \times A_r \to A_r, (a,b) \mapsto a \underset{A^o}{\cdot} b = ba
            \]
        \end{enumerate} 
    \end{notation}

    \subsection*{The \(R\)-algebra \(M_n(A)\)}

    \(M_n(A)\) denotes the \(R\)-module of \(n \times n\)-matrices with entries in \(A\).

    Multiplication: \((a_{ij})_{1 \leq i,j\leq n} \cdot (b_j)_{i \leq i,j \leq n} \coloneqq \left( \sum_{k = 1}^n a_{ik}b_{kj} \right)_{1 \leq i,j \leq n}\).
    
    \begin{proposition}
        \begin{enumerate}[label=\roman*)]
            \item The map \(\lambda : A \to \operatorname{End}_R(A), \alpha (a) = [b \mapsto ab]\) is injective and induces an isomorphism of \(R\)-algebras \(A \xrightarrow[\lambda]{\cong}\operatorname{End}_{A^o}(A) \subset \operatorname{End}_R(A)\). 
            \item The map \(\rho : A^o \to \operatorname{End}_R(A), \rho (a) = [b \mapsto ba]\) is injective and induces an isomorphism of \(R\)-algebras \(A^o \xrightarrow[\rho]{\cong} \operatorname{End}_A(A) \subset \operatorname{End}_R(A)\). 
        \end{enumerate} 
    \end{proposition}

    We will discuss this later.

    \begin{proposition}

        \begin{enumerate}[label=\roman*)]
            \item The map \(\operatorname{End}_A(A_{\ell}^{\oplus n}) \xrightarrow{\cong} M_n(A^o)\), given by \(f \mapsto (a_{ij})_{i,j}\) where \(f(e_j) = \sum_{i=1}^n a_{ij}e_i\) is an isomorphism of \(R\)-algebras.
            
            \item The map \(\operatorname{End}_{A^o}(A_r^{\oplus n}) \xrightarrow{\cong} M_n(A)\), given by \(f \mapsto (a_{ij})_{i,j}\) where \(f(e_j) = \sum_{i=1}^n a_{ij}e_i\) is an isomorphism of \(R\)-algebras. 
        \end{enumerate}
    \end{proposition}

    \begin{proof}
        \begin{enumerate}[label=\roman*)]
            \item Suppose \(f \mapsto (a_{ij})_{i,j}, g \mapsto (a_{ij})_{i,j}\). Then \((f \circ g)(e_j) = f(g(e_j)) = f \left( \sum_{i=1}^n b_{ij} e_j \right) = \sum_{i=1}^n b_{ij} f(e_j) = \sum_{i=1}^n b_{ij} \sum_{k=1}^n a_{k_i} e_k = \sum_{k=1}^n \left( \sum_{i=1}^n b_{ij} a_{k_i} \right) e_k = \sum_{k=1}^n \left( \sum_{i=1}^n a_{ki} \underset{A^o}{\cdot} b_{ij} \right) e_k\). Note that \(\sum_{i=1}^n a_{ki} \underset{A^o}{\cdot} b_{ij} \)  is simply the entry in position \((k,j)\) of \((a_{ij}) \cdot (b_{ij})\) where they're elements of \(M_n(A^o)\).
            
            Hence, \(f \circ g \mapsto (a_{ij}) \underset{M_n(A^o)}{\cdot} (b_{ij})\).
            
            \item Is identical.
        \end{enumerate} 

    \end{proof}

    Now we generalize. Let \(N \in A \text{-Mod}\) and consider \(N^n\) [which is a shorthand for \(N^{\oplus n}\)] as an \(A \text{-Mod}\) by diagonal multiplication of \(A\).

    Let \(\iota_j : N \to N^n\) be the inclusion of the \(j\)'th summand, \(x \mapsto (0 , \cdots , 0, x , 0, \cdots , 0)\) where \(x\) is in the \(j\)'th position.

    Let \(\pi_i : N^n \to N\) be the projection onto \(i\)'th direct summand, \((x_1 , \cdots , x_n) \mapsto x_i\).  

    \begin{proposition}
        The map \(\operatorname{End}_A(N^n) \xrightarrow{\cong} M_n(\operatorname{End}_A(N))\) where \(f \mapsto (f_{ij})_{1 \leq i,j \leq n}, f_{ij} = \pi_i \circ f \circ \iota_j\) is an isomorphism of \(R\)-algebras. Concretely, \((x_1 , \cdots , x_n) \mapsto \left(\sum_{j=1}^n f_{ij}(x_j) \right)_{i=1,\cdots ,n}\).

        Moreover setting \(C = \operatorname{End}_A(N), N^{\prime} = N^n, C^{\prime} = \operatorname{End}_A(N^{\prime})\) then the inclusion \(\operatorname{End}_C(N) = \operatorname{End}_{\operatorname{End}_A(N)}(N) \xrightarrow{\Delta} \operatorname{End}_C(N^{\prime}) = \operatorname{End}_{\operatorname{End}_A(N)}(N^n) \cong M_n(\operatorname{End}_C(N))\).

        Here \(\Delta : g \mapsto [(x_1 , \cdots , x_n) \mapsto (g(x_1), \cdots , g(x_n))]\).

        This induces an isomorphism (of \(R\)-algebras) \(\operatorname{End}_C(N) \to \operatorname{End}_{C''C^{\prime}}(N^{\prime}) = \operatorname{End}_{\operatorname{End}_A(N^n)}(N^n)\). 
    \end{proposition}

    Note that \(\operatorname{End}_A(A_\ell) \cong A^o\), so this is sort of a generalization to the previous proposition. 

    \begin{proof}
        The first assertion is an easy exercise.

        Let \(g\in \operatorname{End}_C(N)\) and let \(\widetilde{g}: N^n \to N^n\) be \((x_1 , \cdots , x_n) \mapsto (g(x_1), \cdots , g(x_n))\). We want to show that \(\widetilde{g}\) commutes with elements of \(C^{\prime}\).

        Let \(f\in C^{\prime} = \operatorname{End}_A (N^{\prime})\). 
        
        \(\widetilde{g} (f(x_1, \cdots , x_n)) = \widetilde{g} \left( \left( \sum_{j=1}^n f_{ij}(x_j) \right)\right)_i = \left( g \left( \sum_{j=1}^n f_{ij}(x_j) \right)  \right) _i = \left( \sum_{j} g(f_{ij}(x_j)) \right)_i\)
        
        \(= f \left( (g(x_1), \cdots , g(x_n)) \right)\).
        
        Here \(f_{ij} \in C\).
        
        Surjectivity of \(\operatorname{End}_C(N) \to \operatorname{End}_{C^{\prime}}(N^{\prime})\): given \(h\in \operatorname{End}_{C^{\prime}}(N^{\prime}) [\subset \operatorname{End}_C (N^{\prime}) \cong M_n(\operatorname{End}_C(N))]\), we need to show that it is diagonal. Which means we have to show that \(\forall i \neq j: \pi_i \circ h \circ \iota_j = 0\) and \(\pi_i \circ h \circ \iota_i = \pi_j \circ h \circ \pi_j\).

        \(h \in \operatorname{End}_{C^{\prime}}(N^n) \implies h\) commutes with \(\iota_j \circ \pi_i : N^n \to N^n\). Note that \(\iota_j \circ \pi_i \in C^{\prime} = \operatorname{End}_A(N^n)\).

        Then \(h \circ \iota_j = h \circ \iota_j \circ \pi_i \circ \iota_i\).
        
        Compose with \(\pi_i\) on the left: \(\pi_i \circ h \circ \iota_j = \underbrace{\pi_i \circ \iota_j}_{=0} \omega \pi_i \circ h \circ \iota_i = 0\). 

        Compose with \(\pi_j\) on the left: \(\pi_j \circ h \circ \iota_j = \pi_i \circ h \circ \iota_i\).

    \end{proof}

    \section*{Thursday, 1/15/2026}
    
    \begin{corollary}
        \begin{enumerate}[label=\roman*)]
            \item \(\operatorname{End}_A(A^n) \xrightarrow{\cong} M_n(A^o)\).
            \item \(A \xrightarrow{\cong} \operatorname{End}_{\operatorname{End}_A(A^n)}(A^n)\).
            \item \(A^o \cong \operatorname{End}_{M_n(A)}(A^n)\).
            \item \(Z \left( \operatorname{End}_A(A^n) \right) = Z(A^o) \cdot \operatorname{id}_{A^n} = Z(A) \cdot \operatorname{id}_{A^n}\).
            \item \(Z(M_n(A)) = Z(A) \cdot 1_n\) where \(1_n = n \times n\) identity matrix.
        \end{enumerate} 
    \end{corollary}

    \begin{proof}
        \begin{enumerate}[label=\roman*)]
            \item Follows from 1.7 and 1.5.
            \item Take \(N = A_\ell\) in 1.7 and \(C = \operatorname{End}_A(A) \overset{1.5}{=} A^o\).
            
            \(\operatorname{End}_C(N) = \operatorname{End}_{A^o}(A_\ell) = A \xrightarrow[1.7]{\cong} \operatorname{End}_{\operatorname{End}_A(A^n_\ell)}(A^n_\ell)\).  
        \end{enumerate} 

        iii to v are exercises.
    \end{proof}

    \begin{definition}
        [Anti-isomorphism] \(\alpha\) is an \textit{anti-isomorphism} if it is an isomorphism of \(R\)-modules sending \(1_A\) to \(1_A\) and \(\forall a,b \in A \colon \alpha (ab) = \alpha(b) \alpha(a)\).
    \end{definition}

    \begin{proposition}
        \begin{enumerate}[label=\roman*)]
            \item Supppose \(\exists\) anti-isomorphism \(\alpha : A \to A\). Then \(\alpha\) is an isomorphism \(A \to A^o\).
            \item The map \(M_n(A) \to M_n(A^o)^o\) given by \(x \mapsto x^T\) is an isomorphism of algebras.
            \item If \(A\cong A^o\) [as \(R\)-algebras], then \(M_n(A) \cong M_n(A)^o\). In particular, matrix algebras over fields are isomorphic to their opposite. 
        \end{enumerate} 
    \end{proposition}

    \begin{proof}
        HW1.
    \end{proof}

    \begin{example}
        \begin{enumerate}[label=\arabic*)]
            \item \(\mathbb{H} = \mathbb{R} 1 \oplus \mathbb{R} i \oplus \mathbb{R} j \oplus \mathbb{R} k\) the Hamilton Quaternions are isomorphic to \(\mathbb{H}^o\) has an anti-isomorphism given by \(a + bi + cj + dk \mapsto a - bi - cj - dk\). 
            \item Let \(R\) be a field and \(A = \left\{ \begin{bmatrix}
                a & b \\
                0 & d \\
            \end{bmatrix} : a,b,d\in R \right\}\). Then \(A^o \not\cong A\).
        \end{enumerate} 
    \end{example}

    \begin{definition}
        \begin{enumerate}[label=\arabic*)]
            \item A module \(N\) is called \textit{simple} (or \textit{irreducible}) if \(N\neq 0\) and the only submodules of \(N\) are \(0\) and \(N\).
            \item A module \(N\) is called \textit{semisimple} if it is isomorphic to a direct sum of (not necessarily finitely many) simple modules. Formally, we allow the empty direct sum. As a consequence, \(0\) is a semisimple module. 
            \item A submodule \(N \subset M\) is called \textit{minimal} (resp. \textit{maximal}) if it is minimal (resp. maximal) among all non-zero submodules (resp. among all proper submodules). \(0\) doesn't have any minimal or maximal submodule.
        \end{enumerate} 
    \end{definition}

    \begin{example}
        Let \(R = K\) be a field. The simple \(K\)-modules are just the \(1\)-dimensional vector spaces, and they're all isomorphic to \(K\) itself.

        Recall that every vector space has a basis [assuming axiom of choice]. Therefore, every \(K\)-module is semisimple.

        Same is true for division algebras.
    \end{example}

    \setcounter{theorem}{10}
    % proposition 1.10 missing

    \begin{proposition}
        Let \(M\) be a \textit{finitely generated} (f.g.) \(A\)-module, and \(N_0 \subsetneq M\) a proper submodule. Then \(M\) has a maximal submodule \(N\) containing \(N_0\).
    \end{proposition}

    \begin{proof}
        We use Zorn's Lemma.

        Let \(\mathscr{M} = \{ N \subsetneq M : N \text{ submodule, } N \supseteq N_0 \}\).

        \(N_0 \in \mathscr{M} \implies \mathscr{M} \neq \varnothing\).

        Let \(N_1 \subset N_2 \subset \cdots \subset\) an ascending chain in \(\mathscr{M}\). Define \(\widetilde{N} = \bigcup_{i \geq 1} N_i\).

        Let \(m_1 , m_2 , \cdots , m_r\) be generators of \(M\). If \(\widetilde{N} = M\) then there exists \(i \geq 1\) such that \(m_1 , \cdots , m_r \in N_i\). But that would imply \(N_i = M\). This is a contradiction.

        Now we apply Zorn's Lemma. \(\mathscr{M}\) contains a maximal element \(N\). This \(N\) is maximal among all proper submodules. Therefore, \(N \supset N_0\) is maximal.
    \end{proof}

    \begin{proposition}
        Let \(0\neq N \in A\text{-Mod}\). TFAE:

        \begin{enumerate}[label=\roman*)]
            \item \(N\) is simple.
            \item \(\forall x\in N \setminus \{ 0 \} : Ax = N\).
            \item \(\exists \) maximal (left) ideal \(I \subset A : N \underset{A}{\cong} A / I\). 
        \end{enumerate} 
    \end{proposition}

    \begin{proof}
        Consider the map \(A \twoheadrightarrow N\) given by \(a \mapsto ax\). Let \(I = \operatorname{Ann}_A(x) \xrightarrow{N \text{ simple}} I\) is maximal.
    \end{proof}

    \begin{proposition}
        Let \(n > 0\) and \(V\) be an \(n\)-dimensional vector space over a division algebra \(D\), and set \(A = \operatorname{End}_D(V)\). Regard \(V\) as an \(A\)-module. Then,

        \begin{enumerate}[label=\roman*)]
            \item \(V\) is simple as an \(A\)-module.
            \item \(A_\ell \cong V^n\). In particular \(A_\ell\) is a semisimple \(A\)-module.
        \end{enumerate} 
    \end{proposition}

    \begin{proof}
        \begin{enumerate}[label=\roman*)]
            \item We use the fact that if \(v_1 , \cdots , v_n\) is a basis of \(V\) over \(D\), then the map \(\operatorname{End}_D (V) \to \underbrace{V \oplus \cdots \oplus V}_{n} = V^n\) given by \(a \mapsto (a(v_1), \cdots , a(v_n))\) is a bijection of abelian groups.
            
            It is not necessarily a bijection of \(D\)-vector spaces.
            
            Given any \(v_1 \in V \setminus \{ 0 \} \) and any \(w\in V\), we can extend \(\{ v_1 \}\) to be a basis of \(V\) and find \(a\in A\) such that \(a(v_1) = w\). Then \(Av_1 = V \underset{1.12}{\implies} V\) is simple.

            \item Note that the map \(A = \operatorname{End}_D(V) \to V^n\) in (i) is \(A\)-linear, hence an isomorphism of \(A\)-modules.
        \end{enumerate} 
    \end{proof}

    \begin{remark*}
        Take in 1.13 \(V = D^n\). Then \(\operatorname{End}_D(D^n) \underset{1.8}{\cong} M_n(D^o)\).

        Then, 1.13 says \(M_n(D^o) \cong \underbrace{D^n \oplus \cdots \oplus D^n}_{n}\) where the \(D^n\) are columns of the matrices in the LHS. 
    \end{remark*}

    \begin{definition}
        \begin{enumerate}[label=\arabic*)]
            \item An algebra \(A\) is called \textit{semisimple} if \(A_\ell\) is a semisimple \(A\)-module.
            \item \(A\) is called \textit{simple}, if \(A\neq 0\) and does not contain any (\(2\)-sided) ideals other than \(0\) and \(A\).
        \end{enumerate} 
    \end{definition}

    \begin{example}
        \(M_n(D)\) where \(D\) is a division ring is a semisimple algebra by 1.13. This algebra is also simple.

        Note that not every semisimple algebra is simple. Let \(A\) and \(B\) be semisimple. Exercise: \(A \times B\) with componentwise addition and multiplication is semisimple.

        Let \(K_1 , \cdots , K_n\) be fields (or skew fields). Then \(K_1 \times \cdots \times K_n\) is semisimple.
    \end{example}

    Caution: There are simple algebras which are not semisimple (HW1).

    \begin{proposition}
        [Schur's Lemma]

        Let \(M, N \in A\text{-Mod}\). Set \(H = \operatorname{Hom}_A(M,N)\).

        \begin{enumerate}[label=\roman*)]
            \item If \(M\) is simple, any non-zero \(f\in H\) is injective. If \(N\) is simple, any non-zero \(f\in H\) is surjective.
            \item If \(M, N\) are simple, then \(M\cong N\) or \(H = 0\).
            \item If \(M\) is simple, then \(\operatorname{End}_A(M)\) is a division algebra.
        \end{enumerate} 
    \end{proposition}

    \begin{proof}
        Straightforward.
    \end{proof}

    \begin{lemma}
        Suppose \(M = \sum_{i\in I} N_i\) is the sum of a family \((N_i)_{i\in I}\) of \textit{simple} submodules.
        
        Let \(N \subset M\) be any submodule. Then \(\exists J \subset I\) such that,

        \[
            M = N \oplus \bigoplus_{j\in J} N_j
        \]
    \end{lemma}

    \begin{proof}
        Let \(\mathscr{S} = \left\{ J \subset I : \text{s.t. } M + \sum_{j\in J} N_j = N \oplus \bigoplus_{j\in J} N_j  \right\} \). Note that \(\varnothing \in \mathscr{S} \implies \mathscr{S} \neq \varnothing\).

        If \(J_1 \subset J_2 \subset \cdots\) is a chain in \(\mathscr{S}\) and \(\widetilde{J} = \bigcup_{k=1}^{\infty} J_k\) then [exercise] \(\widetilde{J} \in \mathscr{S}\).

        Zorn's lemma implies that \(\mathscr{S}\) has a maximal element \(J\).

        Check: \(J\) works.

    \end{proof}

    \section*{Tuesday, 1/20/2026}
    
    \begin{proposition}
        For \(M \in A \text{-Mod}\). TFAE:


        \begin{enumerate}[label=\roman*)]
            \item \(M\) is the sum of simple submodules.
            \item \(M\) is semisimple.
            \item Every submodule is a direct summand of \(M\). 
        \end{enumerate} 
    \end{proposition}

    \begin{proof}
        i \(\implies\) ii: take \(N = 0\) in 1.15.

        ii \(\implies\) iii: Apply 1.15 with the submodule as \(N\).

        iii \(\implies\) i: Let \(x\in M \setminus \{ 0 \} \) and consider the `cyclic submodule' \(C = Ax \subset M\). Note that \(x\in C \implies 0\neq C\).

        1.11 implies \(\exists\) maximal submodule \(L \subsetneq C\). Since \(L\) is a maximal submodule, \(C / L\) must be simple.

        By assumption, \(\exists N \subset M\) such that \(M = L \oplus N\). Since \(L \subset C\) it follows that \(C = L \oplus (C \cap N)\).

        Therefore, \(C / L \cong C \cap N \implies C \cap N\) is a simple module.

        It follows that every non-zero submodule of \(M\) contains a simple submodule.

        Set \(M^{\prime} = \sum_{N \subset M \text{ simple}} N\). Let \(M^{\prime\prime} \subset M\) be the summand, i.e. \(M = M^{\prime} \oplus M^{\prime\prime}\).
        
        If \(M^{\prime\prime} \neq 0\), it contains a simple submodule \(N_0 \subseteq M^{\prime\prime}\). But then \(N_0 \subset M^{\prime}\). This is a contradiction.
        
        Thus, \(M^{\prime\prime} = 0\). Which means \(M = M^{\prime} =\) sum of simple submodules.
    \end{proof}

    \begin{remark*}
        Given any \(M \in A\text{-Mod}\), the \textit{socle} of \(M\), \(\operatorname{soc}(M)\) is defined to be the largest semisimple submodule of \(M\).
        
        By 1.16, \(\operatorname{soc}(M) = \sum_{N \subset M \text{ simple}} N\). 
    \end{remark*}

    \begin{proposition}
        Suppose \(M\) is the sum of simple submodules \((N_i)_{i\in I}\). Then,

        \begin{enumerate}[label=\roman*)]
            \item Every submodule or quotient module is isomorphic to \(\bigoplus_{j\in J} N_j\) for some \(J \subset I\). Note that it need not be directly equal, just isomorphic. Consider \(K \xhookrightarrow{\Delta} K \oplus K\).
            \item Any simple submodule \(N \subset M\) is isomorphic to one of the \(N_i\). 
        \end{enumerate} 
    \end{proposition}

    \begin{proof}
        ii follows from i.

        For i: Let \(M \underset{\pi}{\twoheadrightarrow} M^{\prime\prime}\) be a quotient of \(M\). Then, \(\overline{N_i} \coloneqq \pi (N_i)\) is zero or simple.
        
        Set \(\overline{I} \coloneqq \{ i\in I \mid \overline{N_i} \neq 0 \} \).
        
        \(\implies M^{\prime\prime} = \sum_{i\in \overline{I}} \overline{N_i}\). Applying 1.15, \(\exists \overline{J} \subset \overline{I}\) so that \(M^{\prime\prime} = \bigoplus_{j\in \overline{J}} \overline{N_j} \cong \bigoplus_{j\in \overline{J}} N_j\).

        If \(M^{\prime} \subset M\) is a submodule, 1.16 \(\implies \exists N \subset M : M = N \oplus M^{\prime}\). Then \(M^{\prime} \cong M / N\), which is a quotient of \(M\).
    \end{proof}

    \begin{corollary}
        \begin{enumerate}[label=\roman*)]
            \item Every submodule or a quotient module of a semisimple module is semisimple.
            \item If \(A\) is semisimple, any simple \(A\)-module is isomorphic to a submodule of \(A_\ell\), and so is isomorphic to a minimal left ideal of \(A\). 
        \end{enumerate} 
    \end{corollary}

    \begin{proof}
        \begin{enumerate}[label=\roman*)]
            \item Directly follows from 1.17.
            \item  Suppose \(A\) is semisimple. Then by definition, \(A_\ell = \bigoplus_{i\in I} N_i\) where \(N_i\) are simple. If \(N\) is a simple module, 1.12 implies \(N \cong A_\ell / I\) where \(I \subset A\) is a maximal submodule.
            
            1.17 implies that \(N\cong N_i\) for some \(i\).

            Note that each \(N_i\) is a left ideal. Since it is simple, it must be minimal among the non-zero submodules.
        \end{enumerate} 
    \end{proof}

    \begin{proposition}
        For an algebra \(A\) TFAE:

        \begin{enumerate}[label=\roman*)]
            \item \(A\) is semisimple.
            \item Every \(A\)-module is semisimple. 
        \end{enumerate} 
    \end{proposition}

    \begin{proof}
        ii \(\implies\) i is direct.

        i \(\implies\) ii: Suppose \((m_i)_{i\in I}\) is a set of generators of \(M\) as an \(A\)-module. Consider \(A_\ell^{(I)} \coloneqq \bigoplus_{i\in I} A_\ell \twoheadrightarrow M\) given by \((a_i)_{i\in I} \mapsto \sum_{i} a_i m_i\).

        Thus \(M\) must be a quotient of a semisimple module. The statement follows from 1.18.
    \end{proof}

    \begin{definition}
        Denote by \(\mathcal{T}(A)\) the set of isomorphism classes of simple \(A\)-modules.

        We call a simple module \(N\) of type \(\tau \in \mathcal{T}(A)\) if the isomorphism class of \(N\) is \(\tau\).

        Given \(M\in A\text{-Mod}, \tau \in \mathcal{T}(A)\), we set \(M_{\tau} = \sum_{N \subset M \text{ simple of type } \tau} N\) [note that it must be a direct sum of some submodules of type \(\tau\).] \(M_{\tau} \) is called the \textit{\(\tau\)-isotypic component}.

        If \(M = M_{\tau}\) then we say that \(M\) is \textit{isotypic of type \(\tau\).} 
    \end{definition}

    \begin{remark*}
        By 1.12, \(\left\{ I \subset A_\ell \mid \text{maximal left ideal} \right\} \to \mathcal{T}(A)\) given by \(I \mapsto \text{isomorphism class of } A / I\) is surjective. Thus \(\mathcal{T}(A)\) is a set.
        
        Exercise: This map is bijective (HW2).
    \end{remark*}

    \begin{notation*}
        \(A\text{-Mod}^{\text{ss}} =\) category of semisimple \(A\)-modules.
    \end{notation*}

    \begin{corollary}
        Let \(M \in A\text{-Mod}^{\text{ss}}\). Then,
        
        \begin{enumerate}[label=\roman*)]
            \item \(M = \bigoplus_{\tau \in \mathcal{T}(A)} M_{\tau}\).
            \item \(\forall\) submodule \(M^{\prime} \subset M: M^{\prime} = \bigoplus_{\tau \in \mathcal{T}(A)} (M^{\prime} \cap M_{\tau})\).   
        \end{enumerate} 
    \end{corollary}

    \begin{proof}
        \begin{enumerate}[label=\roman*)]
            \item \(M\) semisimple \(\implies M = \sum_{\tau} M_{\tau}\). Fix \(\tau\). 1.18 implies that \(M^{\prime} \coloneqq M_{\tau} \cap \left( \sum_{\tau^{\prime} \neq \tau} M_{\tau^{\prime}} \right)\) is semisimple.
            
            If \(M^{\prime} \neq 0\) then \(\exists N \subset M^{\prime}\) which is simple. \(N \subset M_\tau\) implies \(N\) must be of type \(\tau\). However, \(N \subset \sum_{\tau^{\prime} \neq \tau} M_{\tau^{\prime}}\),, which implies that \(N\) must not be of type \(\tau\). This is a contradiction. Therefore, \(M^{\prime} = 0\).

            This means the sum is direct, i.e. \(M^{\prime} = \bigoplus_{\tau} M_{\tau}\).

            \item 1.18 implies that \(M^{\prime}\) must be semisimple. Therefore \(M^{\prime} = \bigoplus_{\tau} M^{\prime}_{\tau}\).
            
            Clearly \(M_{\tau}^\prime \subset M^{\prime} \cap M_{\tau}\).

            Conversely, \(M^{\prime} \cap M_{\tau}\) is semisimple by 1.18. Therefore, \(M^{\prime} \cap M_{\tau} = \sum_{N \subset M^{\prime} \cap M_{\tau}, N \text{ simple} } N\). Note that each \(N\) is of type \(\tau\). Thus \(M^{\prime} \cap M_\tau = M^\prime_{\tau}\).
        \end{enumerate} 
    \end{proof}

    \begin{proposition}
        If \(M, M^{\prime} \in A \text{-Mod}^{\text{ss}}\), then \(\operatorname{Hom}_A (M,M^{\prime}) = \coprod_{\tau \in \mathcal{T}(A)} \operatorname{Hom}_A(M_{\tau}, M_{\tau}')\). 
    \end{proposition}

    \begin{proof}
        Suppose \(f\in \operatorname{Hom}_A(M, M^{\prime})\). Consider \(\eval{f}_{M_{\tau}}: M_{\tau} \twoheadrightarrow f(M_{\tau}) \subset M^{\prime}\).
        
        Note that \(f(M_{\tau}) = \sum_{N \subset M \text{ simple of type } \tau} f(N)\). Thus \(f(M_{\tau}) \subset M^{\prime}_{\tau}\).  
    \end{proof}

    \begin{proposition}
        \(M \in A \text{-Mod}^{ss}\). For a submodule \(U \subset M\) TFAE:

        \begin{enumerate}[label=\roman*)]
            \item \(\forall f\in \operatorname{End}_A(M), f(U) \subset U\).
            \item \(U\) is a direct sum of some \(M_{\tau}, \tau \in \mathcal{T}(A)\).  
        \end{enumerate}
        
        \begin{proof}
            ii \(\implies\) i is a direct consequence of 1.21.

            i \(\implies\) ii: 1.20 and 1.18 imply \(U = \bigoplus_{\tau} U_{\tau}\).

            If \(U_{\tau} \subsetneq M_{\tau}\), 1.15 \(\implies M_{\tau} = U_{\tau} \bigoplus_{i\in I} N_i\) where the sum is non-zero and each \(N_i\) is of type \(\tau\).

            Then one finds \(f: U \to M\) such that \(f(U) \not\subset U\). Can extend \(f\) to a homomorphism \(f: M \to M\). 
        \end{proof}
    \end{proposition}

    \section*{Thursday, 1/22/2026}
    
    \begin{corollary}
        Let \(A\) be semisimple. For a subset \(U \subseteq A\) TFAE:

        \begin{enumerate}[label=\roman*)]
            \item \(U\) is an ideal of \(A\). (By convention we mean a two-sided ideal). 
            \item \(U\) is a direct sum of isotypic components of \(A_\ell\).
        \end{enumerate} 
    \end{corollary}

    \begin{proof}
        i \(\implies\) ii: \(U\) is an ideal \(\implies U\) is a submodule of \(A_\ell\), and \(U\) is stable under `right multiplication by \(a\)'. We denote this right multiplication by \(a\) map to be \(_A a : A_\ell \to A_\ell\), \(_A a (b) = ba\). So \(U\) is stable under all \(_A a\).
        
        1.5 \(\implies \operatorname{End}_A(A_\ell) = A^o\).
        
        Therefore, \(\forall f\in \operatorname{End}_A(A_\ell) : f(U) \subset U\). Statement ii follows from 1.22.

        ii \(\impliedby\) i: same argument in reverse order.
    \end{proof}

    \begin{corollary}
        Let \(A\) be semisimple.

        \begin{enumerate}[label=\roman*)]
            \item The isotypic components of \(A_\ell\) are precisely the minimal ideals of \(A\), and every ideal is a direct sum of minimal ideals.
            \item \(A\) has only finitely many minimal ideals. 
            \item \(\vert \mathcal{T}(A) \vert < \infty\).
        \end{enumerate} 
    \end{corollary}

    \begin{proof}
        \begin{enumerate}[label=\roman*)]
            \item \(I \subseteq A_\ell\) a direct sum of isotypic components \(\underset{1.23}{\iff} I\) is an ideal.
            
            Therefore, \(I\) is an isotypic component \(\iff\) \(I\) is a minimal ideal.

            \item Write \(A = \bigoplus_{\tau \in \mathcal{T}(A)} (A_\ell)_{\tau}\). Then we can write \(1 = \sum_{\tau} a_{\tau} \) where \(a_{\tau} \in (A_{\ell})_{\tau}\) and all but finitely many \(a_\tau\) are zero.
            
            Then, \(A = \sum_{\tau, a_{\tau} \neq 0} A a_{\tau} \subseteq \bigoplus_{a_{\tau} \neq 0} (A_\ell)_{\tau} \subseteq A\).

            \item 1.18: any simple \(A\)-module \(N\) is isomorphic to a submodule of \(A_\ell\). Suppose \(\tau\) is the type of \(N\). Then, \((A_\ell)_{\tau} \supseteq N\).
            
            Since there are only finitely many isotypic components, there are only finitely many \(\tau \in \mathcal{T}(A)\).
        \end{enumerate} 
    \end{proof}

    \begin{example*}
        \begin{enumerate}[label=\arabic*)]
            \item Let \(K_1 , \cdots , K_n\) be fields. Then, \(A = K_1 \times \cdots \times K_n\) is semisimple, and \(K_i = \{ 0 \} \times \cdots \times \{ 0 \} \times K_i \times \{ 0 \} \times \cdots \times \{ 0 \} \) is a simple module, and these exhaust all of \(\mathcal{T}(A)\).  
            
            \item Suppose \(K\) is a field, and let \(A = M_n(K)\). We can write \(A_\ell\) as a direct sum of columns, i.e. \(A_\ell = Ae_1 \oplus \cdots \oplus Ae_n\).
            
            Here, \(A_{e_i} \underset{A}{\cong} K^n\) is up to isomorphism the only simple \(A\)-module. Denote by \(\tau\) this type. \((A_\ell)_{\tau}\) is the only isotypic component.

            \item \(A = M_{n_1} (K_1) \times \cdots \times M_{n_s}(K_s)\) has \(s\) isotypic components.
        \end{enumerate} 
    \end{example*}

    \begin{lemma}
        Let \(M \in A\text{-Mod}^{\text{ss}}\) and suppose \(M = \bigoplus_{i\in I} N_i\) and \(M = \bigoplus_{j\in J} N_j^{\prime}\) with simple submodules \(N_i , N_j^{\prime} \subset M\).
        
        Then \(\exists\) bijection \(\sigma : I \to J\) such that \(\forall i\in I : N^{\prime}_{\sigma(i)} \underset{A}{\cong} N_i\). In particular, \(I\) and \(J\) must have the same cardinality.
    \end{lemma}

    \begin{proof}
        WLOG we may assume \(M = M_{\tau}\). Let \(N\) be a fixed simple module of type \(\tau\).

        Then \(N \cong N_i \cong N_j^{\prime}\) for all \(i\in I, j \in J\).

        Put \(D = \operatorname{End}_A(N)^o\). 1.44 \(\implies D\) is a division algebra.

        Then, the map \(D \times \operatorname{Hom}_A(N,M) \to \operatorname{Hom}_A(N,M)\) given by \((d,f) \mapsto f \circ d\) makes \(\operatorname{Hom}_A(N,M)\) into a \(D\)-module.
        
        Moreover, \(\operatorname{Hom}_A(N, M) = \bigoplus_{i\in I} \operatorname{Hom}_A (N, N_i)\).
        
        However, \(\operatorname{Hom}_A(N, N_i) \cong \operatorname{Hom}_A(N,N) = \operatorname{End}_A(N) \cong D^o\). Therefore, \(\operatorname{Hom}_A(N, N_i)\) is free of rank \(1\) over \(D\).
        
        Then, \(\dim_D \operatorname{Hom}_A(N, M) = \vert I \vert\).

        By the same argument, \(\dim_D \operatorname{Hom}_A(N, M) = \vert J \vert\) using the other decomposition.
    \end{proof}

    \begin{definition}
        If we write \(M\in A\text{-Mod}^{\text{ss}}\) as \(M = \bigoplus_{i \in I} N_i\) where \(N_i\) are simple, then \(\ell_A(M) = \vert I \vert \) is called the \textit{length} of \(M\) (as an \(A\)-module).
        
        The \textit{length of a semisimple algebra \(A\)} is by definition \(\ell_A(A_\ell)\).

        Given a simple module \(N\) of type \(\tau\) we also use the notation:

        \[
            M : \tau = M : N \coloneqq \ell_A(M_{\tau})
        \]
    \end{definition}

    \begin{remark*}
        If \(N\) is simple and \(D = \operatorname{End}_A(N)^o\), then \(M : N = \dim_D \operatorname{Hom}_A(N,M)\) and \(\ell_A(M) = \sum_{\tau \in \mathcal{T} (A)} M : \tau\) 
    \end{remark*}

    \begin{theorem}
        [Jacobson Density Theorem]

        Let \(M\in A\text{-Mod}^{\text{ss}}\) and \(C = \operatorname{End}_A(M)\) and consider the canonical map:

        \begin{enumerate}[label=(\arabic*)]
            \item \(A \to \operatorname{End}_C(M), a \mapsto a_M = [m \mapsto am]\). Note that this map is not necessarily \(A\)-linear. But it is \(C\)-linear.
        \end{enumerate} 

        Then, the image of (1) is `dense' in \(\operatorname{End}_C(M)\) in the following sense:

        \(\forall f\in \operatorname{End}_C(M)\) and \(\forall x_1 , \cdots , x_n \in M, \exists a \in A\) such that \(\forall 1 \leq i \leq n : f(x_i) = a \cdot x_i \quad (2)\).

        If \(M\) is finitely generated as a \(C\)-module, the map (1) is surjective.
    \end{theorem}

    \begin{proof}
        Consider \(M^{\prime} \coloneqq M^{\oplus n} \in A\text{-Mod}^{\text{ss}}\) and \(x \coloneqq (x_1 , \cdots , x_n) \in M^{\prime}\).
        
        1.16 \(\implies \exists M^{\prime\prime} \subset M^{\prime}\) with the property \(M^{\prime} = M^{\prime\prime} \oplus Ax\).

        Then we can define a projection \(p\) as follows: \(p: M^{\prime} \to M^{\prime}\) as follows: every element of \(M^{\prime} = M^{\prime\prime} + Ax\) can be written as \(m^{\prime\prime} + ax\) where \(m^{\prime\prime} \in M^{\prime\prime}\). Then, \(p(m^{\prime\prime} + ax) \coloneqq ax \in M^{\prime}\).

        Then, \(p\in \operatorname{End}_A(M^{\prime}) \eqqcolon C^{\prime}\).
        
        Now define \(\widetilde{f}: M^{\prime} \to M^{\prime}\) as follows: \(\widetilde{f}(y_1 , \cdots , y_n) = (f(y_1) , \cdots , f(y_n))\).

        1.7 \(\implies \operatorname{End}_C (M) \xrightarrow{\cong} \operatorname{End}_{C^{\prime}}(M^{\prime})\) given by \(f \mapsto \widetilde{f}\) is a \(C^{\prime}\)-linear bijection.

        Thus, \(\widetilde{f} \circ p = p \circ \widetilde{f}\).

        Therefore, \((f(x_1), \cdots , f(x_n)) = \widetilde{f}(x) = \widetilde{f}(p(x)) = p(\widetilde{f}(x)) = a \cdot x\) for some \(a\in A\) [since \(\operatorname{im} p = Ax\)].

        Therefore, \(\forall 1 \leq i \leq n : f(x_i) = a x_i\).
    \end{proof}

    \begin{corollary}
        Let \(N\in A\text{-Mod}\) be simple and \(D = \operatorname{End}_A(N)\). Consider \(N\) as a \(D\)-vector space. Let \(x_1 , \cdots , x_n\) be linearly independent over \(D\). Then for any \(y_1 , \cdots , y_n \in N \exists a \in A \forall 1 \leq i \leq n : y_i = a x_i\). 
    \end{corollary}

    \begin{proof}
        We apply 1.26 with \(C = \operatorname{End}_A(N) = D\).

        We extend \((x_1 , \cdots , x_n)\) to a \(D\)-basis of \(N\). Then we can choose \(f \in \operatorname{End}_{D = C} (N)\) such that \(f(x_i) = y_i\) for all \(1 \leq i \leq n\).
    \end{proof}

    \section*{Tuesday, 1/27/2026}
    
    \subsection*{Noetherian and Artinian Modules (E. Artin, 1898-1962, IU 1938-1946)}

    \begin{definition}
        A module \(M\) is called:

        \begin{enumerate}[label=\roman*)]
            \item \textit{Noetherian} if every non-empty subset of submodules of \(M\) has a maximal element.
            \item \textit{Artinian} if every non-empty subset of submodules of \(M\) has a minimal element. 
        \end{enumerate} 

        An algebra \(A\) is called \textit{artinian} (resp. \textit{artinian}) if \(A_\ell\) is noetherian (resp. artinian).
    \end{definition}

    \begin{remark}
        \begin{enumerate}[label=\roman*)]
            \item \(M\) is noetherian (resp. artinian) iff every increasing (resp. decreasing) (countable) chain is stationary (i.e. all members are the same from some index on).
            \item For a division ring \(D\) and \(V\) a \(D\)-Mod, TFAE:
            
            \(V\) is noetherian \(\iff V\) is artinian \(\iff \dim_D(V) < \infty\).

            \item If \(M = \bigoplus_{i\in I} M_i\) with all \(M_i\) non-zero and \(\vert I \vert = \infty\), then \(M\) is neither noetherian nnor artinian.
            \item \(\mathbb{Z}\) is not artinia: \((2) \supsetneq (6) \supsetneq (24) \supsetneq (120) \supsetneq \cdots\). \(\mathbb{Z}\) is noetherian, as is any PID.
            \item \(\mathbb{Q} / \mathbb{Z}\) (as a \(\mathbb{Z}\)-module) is not noetherian: \(\frac{1}{2}\mathbb{Z} / \mathbb{Z} \subsetneq \frac{1}{6}\mathbb{Z} / \mathbb{Z}  \subsetneq \frac{1}{24} \mathbb{Z} / \mathbb{Z} \subsetneq \cdots\).
            
            \(\mathbb{Q} / \mathbb{Z}\) is also not artinian: \(M = \sum_{p \text{ prime}} \frac{1}{p}\mathbb{Z} / \mathbb{Z} \supsetneq \sum_{p > 2} \frac{1}{p} \mathbb{Z}  / \mathbb{Z} \supsetneq \sum_{p > 3} \frac{1}{p}\mathbb{Z} / \mathbb{Z} \supsetneq \cdots\).

            But for \(\mathbb{Z}_{(p)}\) [the localization of \(\mathbb{Z}\) at \(p\)], the module \(\mathbb{Q} / \mathbb{Z}_{(p)}\) is artinian.

            \item If \(K\) is a field, then any finite dimensional \(K\)-algebra \(A\) is noetherian and artinian.
            \item \(\exists\) noetoherian and artiniain \(K\)-algebra \(A\) such that \(A^o\) is neither noetherian nor artinian.
            \item We'll prove: \(A\) artinian \(\implies A\) is noetherian. 
        \end{enumerate} 
    \end{remark}

    \begin{proposition}
        \(M\) is noetherian iff every submodule is finitely generated.
    \end{proposition}

    \begin{proof}
        \(\implies\): given \(N \subset M\) set \(S = \{ N^{\prime} \subset N \mid N^{\prime} \text{ is f.g.} \} \).

        \(0 \in S \implies S \neq \varnothing\), and by assumption \(S\) has a maximal element \(N_0 \subseteq N\). It is easy t show that \(N_0 = N \implies N\) is f.g.

        \(\impliedby\): Let \(N_1 \subseteq N_2 \subseteq \cdots\) be submodules of \(M\). By assumption we can show that \(N = \bigcup_{i} N_i\) is finiely generated. Let \(m_1 , \cdots , m_s\) be generators of \(N\). Then \(\exists i_0: m_1 , \cdots , m_s \in N_{i_0}\). Thus \(N = N_{i_0} = N_{i_0 + 1} = \cdots\). 
    \end{proof}

    \begin{remark*}
        For \(M\) to be noetherian it does not suffice that \(M\) is finitely generated.

        If \(A\) is noetherian and \(M\) is finitely generated, then \(M\) is noetherian.
    \end{remark*}

    \begin{proposition}
        Let \(M\) be a module and \(N \subset M\) be a submodule.

        \begin{enumerate}[label=\roman*)]
            \item \(N\) and \(M / N\) are noetherian \(\iff M\) is noetherian.
            \item \(N\) and \(M / N\) are artinian \(\iff\) \(M\) is artinian. 
        \end{enumerate} 
    \end{proposition}

    \begin{proof}
        HW3
    \end{proof}

    \begin{corollary}
        Let \(p\) be the property `noetherian' or `artinian'. 

        \begin{enumerate}[label=\roman*)]
            \item Suppose \(M = \bigoplus_{i=1}^n M_i\). Then \(M\) is \(p\) iff all \(M_i\) are \(p\).
            \item If an algebra \(A\) is \(p\), every \(M \in A\text{-Mod}^{\text{f.g.}}\) is \(p\).
        \end{enumerate} 
    \end{corollary}

    Suppose,

    \[
        0 \to N \hookrightarrow M \to M / N \to 0
    \]

    say: \(M\) is an extension of \(M / N\) by \(N\).

    \begin{proof}
        \begin{enumerate}[label=\roman*)]
            \item Reduce to the case \(n = 2\) and apply 1.30.
            \item \(A\) is \(p \implies A_\ell\) is \(p \xrightarrow{\text{(i)}} A_\ell^{\oplus n}\) is \(p. 1.30 \implies A_\ell^{\oplus n} / L\) is \(p\) for any submodule \(L \subset A_\ell^{\oplus n}\). Any finitely generated module is of this form which implies the assertion.  
        \end{enumerate} 
    \end{proof}

    \begin{proposition}
        Let \(M \in A\text{-Mod}^{\text{ss}}\). TFAE:

        \begin{enumerate}[label=\roman*)]
            \item \(M\) is finitely generated.
            \item \(M\) is a finite direct sum of simple submodules.
            \item \(M\) is artinian.
            \item \(M\) is noetherian. 
        \end{enumerate} 
    \end{proposition}

    \begin{proof}
        Write \(M = \bigoplus_{i\in I} N_i\) where \(N_i\) are simple. Assume i, and let \(m_1 , \cdots , m_s\) be generators. \(\forall j : 1 \leq j \leq s \exists I_j \subset I\) such that \(\vert I_j \vert < \infty\) such that \(m_j \in \bigoplus_{i\in I_j} N_i\).
        
        Then, \(m_1 , \cdots , m_s \in \bigoplus_{i\in \bigcup_{j=1}^s I_j} N_i = M\) which implies ii.
        
        ii \(\implies\) iii, iv is implied by 1.31.

        iv \(\implies\) i: Noetherian implies finitely generated.

        To complete the full circle, assume iii. Use remark 1.28. HW3.
    \end{proof}

    \begin{corollary}
        Every semisimple algebra \(A\) is noetherian and artinian.
    \end{corollary}

    \begin{proof}
        \(A_\ell\) is semisimple and f.g., \(A_\ell = A \cdot 1_A\). Apply 1.32.
    \end{proof}

    \section*{Thursday, 1/29/2026}
    
    \begin{remark*}
        The study of Semisimple Algebras has very little overlap with Commutative Algebras, namely finite product of fields.

        There is a generalization: \textit{Azumaya algebras}.

        Goro Azumaya: IU (1968-1992)
    \end{remark*}

    \begin{proposition}
        [Definition of Composition Series]
        For \(M\in A\text{-Mod}\) TFAE:

        \begin{enumerate}[label=\roman*)]
            \item \(M\) is both artinian and noetherian.
            \item \(M\) has a \textit{composition series}: a chain of submodules:
            
            \[
                M = M_n \supsetneq M_{n-1} \supsetneq \cdots \supsetneq M_0 \coloneqq 0
            \]

            where \(M_i / M_{i-1}\) is simple for \(1 \leq i \leq n [n=0 \text{ if } M = 0]\).

            \(n\) is called the \textit{length of the composition series}.
        \end{enumerate} 
    \end{proposition}

    \begin{proof}
        ii \(\implies\) i: Suppose \(M\) has a composition series of some length. Then \(M_1\) must be simple. If there is an \(M_2\), \(M_2 / M_1\) must be simple. We have the exact sequence:

        \[
            0 \to M_1 \to M_2 \to M_2 / M_1
        \]

        1.30 then implies that \(M_2\) is noetherian and artinian. The fact that \(M\) must be both noetherian and artinian follows from induction.

        i \(\implies\) ii: HW3.
    \end{proof}

    \begin{theorem}
        [Jordan-H\"older Theorem for Modules]

        If \(M\) has to composition series:

        \begin{enumerate}[label=\arabic*)]
            \item \(M = M_n \supsetneq M_{n-1} \supsetneq \cdots \supsetneq M_0 = 0\).
            \item \(M = L_m \supsetneq L_{m-1} \supsetneq \cdots \supsetneq L_0 = 0\).
        \end{enumerate} 

        then \(m = n\). Furthermore, the simple quotient are the same under permutation: \(\exists \sigma \in \mathcal{S}_n\) such that \(\forall 1 \leq i \leq n: L_i / L_{i-1} \underset{A}{\cong} M_{\sigma(i)} / M_{\sigma(i) - 1}\).
    \end{theorem}

    \begin{definition}
        A module \(M\) is called of \textit{finite length} if \(M\) satisfies the equivalent conditions of 1.34.

        The length of any composition series is called \textit{the length of} \(M\), denoted by \(\ell(M)\).
    \end{definition}

    \begin{remark*}
        \begin{enumerate}[label=\roman*)]
            \item If \(M\) is of finite length and \(N\) is a submodule, then \(N\) is of finite length. Furthermore,
            
            \[
                \ell(M) = \ell(N) + \ell(M / N).
            \]

            We use the exact sequence:

            \[
                0 \to N \to M \to M / N.
            \]

            \item \(M\) semisimple of finite length, i.e. \(M = \bigoplus_{i=1}^n N_i\) where \(N_i\) are simple \(\implies \ell(M) = n\).
            
            \(M\) is not always the direct sum of the simple quotients. For example, suppose \(K\) is a field and \(A = \left\{ n \times n \text{ upper triangular \(K\)-matrices} \right\} \) acting on \(K^n\). Then \(Ke_1\) is a submodule but \(Ke_2\) is not one.
            
            \[
                K^n \supsetneq Ke_1 \oplus \cdots Ke_{n-1} \supsetneq Ke_1 \oplus \cdots \oplus Ke_{n-2} \supsetneq \cdots \supsetneq Ke_1 \supsetneq 0.
            \]

            Hence \(\ell(K^n) = n\).
        \end{enumerate} 
    \end{remark*}

    \begin{definition}
        A module \(N\) is called \textit{indecomposable} if \(N\neq 0\) and \(N\) has no direct summand other than \(0\) and \(N\).

        i.e. We cannot write \(N = A \oplus B\) where \(A, B\) are non-zero submodules. For example, \(K^n\) in the previous example is indecomposable as an \(A\)-module.
    \end{definition}

    \begin{example}
        \begin{enumerate}[label=\roman*)]
            \item Again, suppose \(K\) is a field and \(A = \left\{ n \times n \text{ upper triangular \(K\)-matrices} \right\} \) acting on \(K^n\). Then \(K^n\) is indecomposable as an \(A\)-module.
            \item Any simple module is indecomposable.
            \item Suppose \(K\) is a field, and \(A = K[x]\) acting on \(K^n\) where \(x \mapsto \begin{bmatrix}
                0 & 1 & 0 & 0 & 0  \\
                 & \ddots & \ddots & \ddots & 0 \\
                 &  & \ddots & \ddots & 0 \\
                 &  &  & \ddots & 1 \\
                 &  &  &  & 0 \\
            \end{bmatrix} \) [jordan block]. Then \(K^n\) is indecomposable as an \(A\)-module.
        \end{enumerate} 
    \end{example}

    \begin{proposition}
        If \(M\) is artinian or noetherian, then it is a direct sum of finitely many indecomposable modules.
    \end{proposition}

    \begin{proof}
        Assume \(M\) is artinian. Suppose \(M\neq 0\) and \(M\) is decomposable. Then, \(\exists\) non-zero submodules \(M_1, M_2\) such that \(M = M_1 \oplus M_2\).
        
        Consider \(S(M) = \left\{ 0 \subsetneq N \subsetneq M : N \text{ is a direct summand of } M \right\} \) \(M_1 \in S(M) \implies S(M) \neq \varnothing\).

        \(M\) is artinian. Therefore, \(S(M)\) must have a minimal element \(N_0\). \(N_0\) must be indecomposable, since any non-trivial decomposition of \(N_0\) contraditcs its minimality.

        Write \(M = N_0 \oplus M_0^{\prime}\). 

        Consider \(S^{\prime} = \left\{ M^{\prime} \subset M : \exists M_0 \subset M : M = M_0 \oplus M^{\prime} \text{ and } M_0 \text{ is a finite direct sum of indecomposables}  \right\} \).

        Then, \(M^{\prime}_0 \in S^{\prime}\). Thus, \(S^{\prime} \neq \varnothing\). Therefore \(S^{\prime}\) contains a minimal element which we claim is \(M^{\prime}\).

        Suppose \(M^{\prime} \neq 0\). If \(M^{\prime}\) is indecomposable then \(M = (M_0 \oplus M^{\prime}) \oplus 0\) which contradicts the minimality of \(M^{\prime}\), since \(0 \in  S^{\prime}\).
        
        Hence \(M^{\prime}\) is indecomposable. Therefore, \(S(M^{\prime}) \neq \varnothing\). \(M^{\prime}\) must also be artinian, therefore \(S(M^{\prime})\) has minimal element \(N^{\prime}_0\) which is indecomposable. We write \(M^{\prime} = N^{\prime}_0 \oplus M^{\prime\prime} \implies M = (M_0 \oplus N^{\prime}_0) \oplus M^{\prime\prime}\) where \(M^{\prime\prime}\) is minimal in \(S^{\prime}\), which is a contradiction.

        Noetherian case: HW3.
    \end{proof}

    \begin{lemma}
        Let \(M \in A\text{-Mod}\) and \(g\in \operatorname{End}_A(M)\).

        \begin{enumerate}[label=\roman*)]
            \item If \(g\) is surjective and \(M\) is noetherian \(\implies g\) is injective.
            \item If \(g\) is injective and \(M\) is artinian \(\implies g\) is surjective.
        \end{enumerate} 
    \end{lemma}

    \begin{proof}
        HW3.
    \end{proof}

    \begin{lemma}
        Let \(M\) be of finite length and \(f\in \operatorname{End}_A(M)\). Then \(\exists\) decomposition \(M = U \oplus N\) such that \(f(U) \subset U, f(N) \subset N\) and \(\eval{f}_{U}\) is bijective and \(\eval{f}_{N}\) is nilpotent, \((\eval{f}_{N})^k = 0\). 
    \end{lemma}

    \begin{proof}
        HW3.
    \end{proof}

    \begin{lemma}
        Let \(M\) be of finite length and indecomposable. Then every endomorphism \(f\in C \coloneqq \operatorname{End}_A(M)\) is either bijective or nilpotent.

        Moreover, \(I = \left\{ f\in C : f \text{ is nilpotent} \right\} \) is the unique maximal (\(2\)-sided) ideal of \(C\).
    \end{lemma}

    \begin{proof}
        Suppose \(f\) is not bijective. 1.39 implies that \(M = U \oplus N\) where \(f(U) \subset U, f(N) \subset N\), \(\eval{f}_{U}\) is bijective, \(\eval{f}_{N}\) is nilpotent. Since \(f\) is not bijective, \(M\neq U\). Therefore, \(N \neq 0\). Since \(M\) must be indecomposable, \(U = 0\) and \(M = N\). Therefore, \(f\) must be nilpotent.

        Now we prove that \(I\) must be an ideal.

        \(f\in I, h\in C \implies \ker (h \circ  f) \neq 0 \implies h \circ f \in I\).

        \(f\in I, h\in C \implies \operatorname{im} (f \circ h) \subsetneq M \implies f \circ h\) is not bijective \(\implies f \circ h \in I\). 

        We need to show \(I\) is closed under addition. Suppose \(f,g \in I\) but \(f+g\notin I\). Then \(f+g\) must be invertible.

        \(h(f+g)=1 \implies hg = 1-hf\). Since \(hf\) is nilpotent it follows that \(hg\) is invertible, which is a contradiction.
    \end{proof}

    \begin{theorem}
        [Krull-Remak-Schmidt] Let \(M\) be a module of finite length. Suppose \(M = N_1 \oplus \cdots \oplus N_m = N^{\prime}_1 \oplus \cdots \oplus N^{\prime}_n\) with indecomposable submodules \(N_i, N_j^{\prime}\). Then \(m = n\) and \(\exists \sigma \in \mathcal{S}_n\) such that \(\forall 1 \leq i \leq n : N_i^{\prime} \cong N_{\sigma(i)}\).
    \end{theorem}

    \begin{proof}
        By induction on \(\ell(M)\). If \(\ell(M) = 1\) there is nothing to show. Let \(\ell(M) > 1\) and suppose the sttement is true for all modules of length \(< \ell(M)\).

        Let \(\iota_j : N_j \hookrightarrow M\) be the inclusion and \(\pi_j: M \to N_j\) the projection onto \(N_j\).

        Let \(p_j = \iota_j \circ \pi_j \in \operatorname{End}_A(M)\). Then \(p_j\) is an idempotent, \(p_j \circ p_j = p_j\) and \(p_j \circ p_i = 0\) if \(i\neq j\). Furthermore \(p_1 + \cdots + p_m = \operatorname{id}_{M}\).

        We can do the same for the other decomposition: we get \(p_j^{\prime}\) where \(p^{\prime}_1 + \cdots + p^{\prime}_n = \operatorname{id}_{M}\) etc.
        
        Consider: \(\operatorname{End}_A(N_1)\ni f_j = \pi_1 \circ p_j^{\prime} \circ \iota_1: N_1 \hookrightarrow M \xrightarrow{\pi_j^{\prime}} N^{\prime}_j \xhookrightarrow{\iota^{\prime}_j} M \to N_1\).
        
        \(f_1 + f_2 + \cdots + f_n = \pi_1 \circ  \underbrace{(p^{\prime}_1 + \cdots +p^{\prime}_n)}_{\operatorname{id}_{M}} \circ \iota_1 = \operatorname{id}_{N_1}\).        

        1.39 implies that \(\exists 1 \leq j \leq n\) such that \(f_j\) is bijective. After renumbering, we may assume that \(f_1\) is bijective.

        Consider \(g=p^{\prime}_1 \circ p_1: M \to M\). Note that \(\pi_1 \circ g = f_1 \circ \pi_1\).


        Set \(h = g + p_2 + \cdots + p_m : M \to M\). Then \(p_1 \circ h = p_1 \circ  g = \iota_1 \circ \pi_1 \circ g = \iota_1 \circ f_1 \circ \pi_1\). We want to show that \(h\) is bijective.

        Suppose \(x\in M\) such that \(h(x) = 0\).

        \(\implies 0 = p_1(h(x)) = p_1(g(x)) = \iota_1(f_1(\pi_1(x))) = 0\). Since \(\iota_1\) and \(f_1\) are injective, it follows that \(\pi_1(x) = 0\). Therefore \(x\in N_2 \oplus \cdots \oplus N_m \eqqcolon M^{\prime}\). Since \(g = p^{\prime}_1 \circ p_1\) it follows that \(g(x) = 0\).

        \(0 = h(x) = g(x) + p_2(x) + \cdots + p_m(x) = x\) where \(g(x) = 0\) and \(p_j(x) \in N_j\). Therefore \(h\) is injective.

        It follows from 1.37 that \(h\) is bijective.

        Since \(h\) is an automorphism and \(\eval{h}_{N_k} = \operatorname{id}_{N_k}\) for \(k \geq 2\) it follows that \(M = N^{\prime}_1 \oplus N_2 \oplus \cdots \oplus N_m\). By quotienting out \(N^{\prime}_1\) it follows that \(N_2 \oplus \cdots \oplus N_m \cong N^{\prime}_2 \oplus \cdots \oplus N^{\prime}_n\). The theorem follows from induction.

    \end{proof}
    
    \section*{Tuesday, 2/3/2026}
    
    \section{Wedderburn Theory}

    Recall: An algebra \(A\) is called \textit{simple} if \(A\) is non-zero and it has no ideals other than \(0\) and \(A\).

    Recall that by ideal we mean ideals that are both left and right ideals. There can be left ideals that are not right ideals and vice versa. Meaning, an algebra \(A\) being simple doesn't necessarily imply that the module \(A_\ell\) is simple.

    \begin{proposition}
        Let \(A\) be simple. TFAE:

        \begin{enumerate}[label=\roman*)]
            \item \(A\) is semisimple.
            \item \(A\) is artinian.
            \item \(A\) possesses a minimal left ideal \(N\). 
        \end{enumerate} 
    \end{proposition}

    Minimal ideals are by convention non-zero. This forces \(A\) to be non-zero from iii as well.

    Recall by saying \(A\) is noetherian/artinian we mean \(A_\ell\) is noetherian/artinian.

    \begin{proof}
        i \(\implies\) ii: Corollary 1.33 (because \(A_\ell\) is semisimple and finitely generated, hence it has finite length).

        ii \(\implies\) iii: Trivial.

        iii \(\implies\) i: Let \(0 \neq N \subset A\) be the minimal left ideal. Then, \(0 \neq NA = \sum_{a\in A} Na\) is a \(2\)-sided ideal. Since \(A\) is simple, \(NA = A\).

        \(N\) minimal left ideal \(\implies N\) is simple as an \(A\)-module. \(Na\) is the image of \(N\) under the \(A\)-module map \(N \to Na, x \mapsto xa\). If \(Na\) is non-zero, it is the non-zero image of a simple module, which implies \(Na\) is simple. Then, \(NA\) is a sum of simple modules: \(A = NA = \sum_{Na\neq 0} Na\) is a sum of simple modules. Therefore, \(1.16 \implies A\) must be semisimple.
    \end{proof}

    \begin{corollary}
        Let \(A\) be simple and semisimple and \(N \subset A\) a minimal left ideal. Then, \(A_\ell \underset{A}{\cong} N^{\oplus m}\) for some \(m > 0\), and \(A_\ell\) is hence isotypic.
    \end{corollary}

    \begin{proof}
        2.1 implies the existence of a minimal ideal \(N\). Then, \(A_\ell = \sum_{a\in A, Na\neq 0} Na\) where \(Na\) are simple. By 1.17, \(A_\ell \cong \bigoplus_{i\in I} Na_i\) for some subset \(\{ a_i : i \in I \} \) of \(A\). WLOG we assume that \(Na_i \neq 0\) for all \(i\in I\).

        Then, the identity \(1_A = \sum_{j\in J \subset I} n_j a_j\) for some finite \(J \subset I\) and \(n_i \in N\). Therefore, \(\forall a\in A: a = \sum_{j\in J} a n_j a_j\). Note that \(a n_j a_j \in Na_j\) since \(N\) is a left ideal. Therefore, \(A = \bigoplus_{i\in J} Na_i\). Therefore \(J = I\). Furthermore, for each \(i\in I\), \(Na_i \cong N\).
        
        Therefore, \(A_\ell \cong \bigoplus_{i\in I} Na_i \cong N^{\oplus \vert I \vert}\).

    \end{proof}

    \begin{proposition}
        Let \(A\) be a non-zero semisimple algebra. TFAE:

        \begin{enumerate}[label=\roman*)]
            \item \(A\) is simple.
            \item \(A_\ell\) is isotypic.
            \item \(\vert \mathcal{T} (A) \vert = 1\). Recall that \(\mathcal{T} (A)\) is the set of isomorphism classes of simple \(A\)-modules.
        \end{enumerate} 
    \end{proposition}

    \begin{proof}
        i \(\implies\) ii follows from 2.2.

        ii \(\implies\) iii follows from 1.18: every simple module is isomorphic to a submodule of \(A_\ell\).
        
        Recall 1.24 which says that minimal ideals are precisely the isotypic components of \(A_\ell\). in fact, \(A_\ell = \bigoplus_{\tau \in \mathcal{T}(A)} (A_\ell)_\tau\).
        
        If \(N\) is simple of type \(\tau \implies (A_\ell)_{\tau} \cong \bigoplus_{i\in I_\tau} N\).
        
        If \(N\) is simple \(A_\ell / L \cong N \to N\) appears as a direct summand in \(A_\ell = \bigoplus_{\tau \in \mathcal{T}(A)} (A_\ell)_{\tau}\). \((A_\ell)\tau \neq 0\) for all \(\tau \in \mathcal{T}(A)\).
        
        iii \(\implies\) ii is trivial.

        ii \(\implies\) i follows from 1.24.

        Summarizing,

        \[
            \begin{tikzcd}[column sep = huge]
                \text{i} \ar[r,Rightarrow, bend left, "2.2"] & \text{ii} \ar[r, Rightarrow, bend left, "1.18"] \ar[l, Rightarrow, bend left, "1.24"] & \text{iii} \ar[l, Rightarrow, bend left, "\text{trivial}"]
            \end{tikzcd}
        \]
    \end{proof}

    \begin{proposition}
        If \(D\) is a division algebra, then \(M_n (D)\) is a simple artinian algebra for any \(n \geq 1\).
    \end{proposition}

    \begin{proof}
        Set \(V = (D^o)^{\oplus n} \implies A \coloneqq \operatorname{End}_{D^o}(V) \cong M_n((D^o)^o) = M_n(D)\).

        By 1.13: \(A_\ell \overset{(\ast)}{=} \underbrace{V \oplus \cdots \oplus V}_{n \text{ copies}}\) and \(V\) is simple as \(A\)-module (1.13). Then \(A_\ell\) is isotypic and seimisimple because of \(\ast\). Then by 2.3 \(A\) is simple. 
    \end{proof}

    Recall: \(Z(A) \coloneqq\) center of \(A\).

    \begin{proposition}
        \(A\) simple \(\implies Z(A)\) is a field.
    \end{proposition}

    \begin{proof}
        Pick non-zero element of the center \(a\in Z(A) \setminus \{ 0 \}\). Note that \(Aa=aA\) must be a non-zero two-sided ideal, and since \(A\) is simple it follows that \(aA = A\). \(aA = A \implies \exists b\in A\) such that \(ab (=ba) = 1\). Therefore \(a\) must have a two-sided inverse.

        Given \(c\in A: (cb - bc) a = c(ba) - (ba) c = c1 - 1c = c-c = 0\). Multiplying by \(b\) on the right, it follows that \((cb - bc) ab = 0 \implies cb - bc = 0  \implies cb = bc\) for all \(c\in A\). Therefore \(b\in Z(A)\).
    \end{proof}


    \begin{theorem}
        A non-zero semisimple algebra \(A\) has only finitely many distinct minimal ideals \(A_1, \cdots , A_n\). Each \(A_i\) is a unital algebra in its own right with the induced addition and multiplication from \(A\). Note that the identity in \(A_i\) is not equal to the identity in \(A\) if \(n > 1\).

        Moreover, \(A = A_1 \times \cdots \times A_n\) is the product of the algebras \(A_1 , \cdots , A_n\) (with componentwise addition and multiplication), and each \(A_i\) is simple and artinian.

        Conversely, if \(A_1 , \cdots , A_n\) are simple artinian algebras, then \(A \coloneqq A_1 \times \cdots \times A_n\) is semisimple and \(A_1, \cdots , A_n\) are precisely the minimal ideals of \(A\).
    \end{theorem}

    Note that direct sum and product carry the same meaning in this case. We generally use direct sums when we're thinking about the object as a module, and products when we're thinking about the object as a ring.

    \begin{proof}
        1.24 \(\implies A = A_1 \oplus \cdots \oplus A_n\) where \(A_1, \cdots , A_n\) are the isotypic components of \(A_\ell\). These are precisely the (2-sided) ideals.

        We also have: \(A_i A_j \subset A_1 \cap A_j = 0\) for \(i\neq j\).

        Write \(1_A = e_1 + \cdots + e_n\) with \(e_i \in A_i\). \(1_A = 1_A \cdot 1_A \implies 1_A = e_1^2 + \cdots + e_n^2\). Taking projections it follows that \(\forall 1 \leq i \leq n: e_i^2 = e_i\).
        
        \(\forall a\in A_i, a = a\cdot 1_A = ae_1 + \cdots + ae_i + \cdots + ae_n = ae_i = e_i a\).
        
        Therefore, \(e_i\) is the identity element in \(A_i\).

        If \(0\neq I \subset A_i\) is an ideal, then \(I\) is an ideal in \(A\). However, since \(A_i\) is minimal, it follows that \(I = A_i\). Therefore \(A_i\) must be simple.

        If \(0 \neq I \subset A_i\) is a left ideal, then \(I\) is a left ideal in \(A\). Since \(A\) is a semisimple algebra \(A\) must be artinian. Thus \(A_i\) must also be artinian.

        The converse is easy to check (HW4)
    \end{proof}

    \begin{corollary}
        Let \(A = A_1 \times \cdots \times A_n\) be a semisimple algebra with simple algebras \(A_1, \cdots , A_n\). Then,
        
        \[
            Z(A) = Z(A_1) \times \cdots \times Z(A_n)
        \]

        is a product of fields, by 2.5. IN particular, \(Z(A)\) is a field if and only if \(A\) is simple.
    \end{corollary}

    \begin{corollary}
        A commutative semisimple algebra \(A\) is a product of finitely many fields: \(A = K_1 \times \cdots \times K_n\).

        These fields are uniquely determined as subsets of \(A\). Namely, they are the minimal ideals of \(A\).
    \end{corollary}

    \begin{remark}
        A commutative artinian algebra \(A\) is semisimple if and only if its nilradical \(\{ a\in A \mid \exists n > 0 : a^n = 0 \} \) is zero.

        \(\implies\) follows from 2.8.

        \(\impliedby\) uses the theory of redical at the end of chapter 28. Direct proof on HW4.
    \end{remark}

    \begin{corollary}
        Let \(K\) be a field. A finite dimensional commutative \(K\)-algebra having no non-zero nilpotent elements is a product of finitely many field extensions \(K_i / K\) with \([K_i : K] < \infty\).
    \end{corollary}

    \begin{proof}
        By the preceding remark, \(A\) is semisimple. By 2.8, it is a product of fields \(K_1, \cdots , K_n\), all of which have finite degree over \(K\).
    \end{proof}

    Now we state a key theorem of this course.

    \begin{theorem}
        [Wedderburn's Theorem]

        An artinian algebra \(A\) is simple if and only if it is isomorphic (as rings) to a matrix algebra \(M_n(D)\) for a division algebra \(D\) and \(n > 0\):

        \[
            A \cong M_n(D)
        \]

        \(D\) and \(n\) are uniquely determined by \(A\).
    \end{theorem}

    \begin{proof}
        By the remark, \(A\) is semisimple. By 2.8 it is a product of fields \(K_1, \cdots , K_n\), all of which have finite degree \(/ K\).
    \end{proof}

    \(A\) semisimple \(\implies A \cong M_{n_1}(D_1) \times \cdots \times M_{n_r}(D_r)\).

    \([D_1] \cdot [D_2] = [D_1 \otimes_K D_2] = [M_n(D_3)] = [D_3]\). 

\end{document}