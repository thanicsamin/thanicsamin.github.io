\documentclass{article}
\usepackage{amsmath, amsthm, amssymb, amsfonts, mathtools, enumitem, stmaryrd,physics, cancel, tikz-cd, graphicx, float, booktabs}
\usetikzlibrary{arrows}
\usepackage{geometry}
    \geometry{
        a4paper,
        left = 20mm,
        top = 20mm,
        right = 20mm,
        bottom = 30mm
    }
\setlength{\parindent}{0pt}
\setlength{\parskip}{\baselineskip}%

\theoremstyle{definition}
\newtheorem{problem}{Problem}
\newtheorem{solution}{Solution}
\newtheorem*{example}{Example}
\newtheorem*{exercise}{Exercise}
\newtheorem*{definition}{Definition}
\newtheorem{theorem}{Theorem}
\newtheorem*{theorem*}{Theorem}
\newtheorem{proposition}[theorem]{Proposition}
\newtheorem*{proposition*}{Proposition}
\newtheorem{lemma}[theorem]{Lemma}
\newtheorem*{lemma*}{Lemma}
\newtheorem{corollary}[theorem]{Corollary}
\newtheorem*{corollary*}{Corollary}
\newtheorem*{remark}{Remark}

\title{Talks}
\author{Thanic Nur Samin \\ Writing about attended talks in Spring 2026}
\date{\vspace{-5ex}}

\begin{document}
    \maketitle

    \section*{Monday, 1/12/2026}

    \section{Ext Duality in derived category \(\mathcal{O}\) and its variants}

    Let \(F\) be any field of char \(0\). Classically \(\mathbb{C}\).

    \(\underline{G} =\) split reductive algebraic group. e.g. \(\underline{G} = \operatorname{GL}_n , \operatorname{SL}_n, \operatorname{Sp}_{2n}\) etc.

    Suppose \(\underline{G} \supset \underline{P}\), where \(\underline{P}\) is the parabolic subgroup, a group consisting of upper triangular submatrices, e.g. \(\begin{bmatrix}
        GL_{n_1} &  & \ast \\
        0 & GL_{n_2} &  \\
        0 & 0 & GL_{n_3} \\
    \end{bmatrix} \) 

    \(\underline{B} \supset \underline{P}\) where \(\underline{B} \) is the Borel subgroup.

    \(\underline{T} \supset \underline{B} \) where \(\underline{T}\) is the maximal split torus. e.g. \(\underline{T} \cong \mathbb{G}_m^d\).

    We take the successive lie algebras to get \(\mathfrak{g} \supset \mathfrak{p} \supset \mathfrak{b} \supset \mathfrak{t} t\).

    \(\mathfrak{t}^{\ast} = \operatorname{Hom}_F(\mathfrak{f}, F)\) weight space.

    \(U = U(\mathfrak{g})\) the universal enveloping algebra. \(U\)-mod is the category of \(U\)-left modiules.

    \subsection{Category \(\mathcal{O}\) and its variants}

    \(M \in U\)-mod, \(M\) is called \underline{ \(\mathfrak{t}\)-split } if it is a direct sum of generalized weight spaces i.e. \(M = \bigoplus_{\lambda \in \mathfrak{t}^{\ast}} M_\lambda^{\infty}\).

    \(M_\lambda^{\infty} = \bigcup_{i > 0} M_\lambda^i\).
    
    \(M_\lambda^i = \{ m \in M \mid \forall x\in \mathfrak{t} : (x - \lambda (x))^i \cdot m = 0 \} \) generalized weight space.

    \(\Pi (M) = \{ \lambda \in \mathfrak{t}^{\ast} \mid M_\lambda^{\infty} \neq 0 \} \) is the set of weights of \(M \supset \Pi (M)_{alg} =\) algebraic weights \(\ni\) \(\lambda\) is algebraic if \(\exists\) algebraic char \(\underline{T} \xrightarrow{\chi} \mathbb{G}_m\),
    
    \(d_{\chi} = \lambda \colon \mathfrak{t} = \operatorname{Lie}(\underline{T}) \to \underline{Lie}(\mathbb{G}_m) = F\).
    
    \begin{definition}
        \(M\in U\)-mod belongs to \(\mathcal{O}^{p, \infty}\) if:

        \begin{enumerate}[label=\arabic*)]
            \item \(M\) is f.g. as \(U\)-module.
            \item \(M\) is \(\mathfrak{t}\)-split
            \item \(\mathfrak{p}\) acts \underline{locally finite}, i.e. \(\forall m\in M, \dim_F(U(\mathfrak{p})\cdot m) < \infty\).  
        \end{enumerate} 

        If 4) \(\Pi (M) = \Pi(M)_{alg}\) then \(M\in \mathcal{O}_{alg}^{\mathfrak{p},\infty}\).

        If 5) \(M = \bigoplus_{\lambda \in \Pi(M)} M_\lambda^i\) then \(M\in \mathcal{O}_{alg}^{\mathfrak{p},i}\)  
    \end{definition}

    \underline{Classically} (meaning in the work of Bernstein-Gelfand-Gelfand, which is why it is often called the BGG category \(\mathcal{O}\)): \(\mathfrak{g}\) is a semisimple complex lie algebra. Then \(\mathfrak{p} = \mathfrak{b}\), the borel subalgebra. Also, the \(i\) in the definition of generalized eigenspace is always \(1\), meaning we just have regular eigenspaces.
    
    Note that, all finite dimensional modules trivially satisfy this definition.

    There are \(\mathfrak{sl}_2\) modules satisfying the definition that are not even weight modules!

    We consider the examples:

    \begin{enumerate}[label=\arabic*)]
        \item Finite dimensional modules: \(\mathcal{O}^{\mathfrak{g}}\) [here \(\mathfrak{p} = \mathfrak{g}\)] \(\subseteq \mathcal{O}^{\mathfrak{p},\infty} \subseteq \mathcal{O}^{\mathfrak{b},\infty} = \mathcal{O}^{\infty}\).
        \item Verma modules: consider the linear form \(\mathfrak{t} \xrightarrow[\text{weight}]{\lambda} F = F_\lambda\). We have \(\mathfrak{b} \twoheadrightarrow \mathfrak{t}\) and \(\mathfrak{b} \to F\). We can consider \(F_\lambda\) as a \(\mathfrak{b}\)-module, i.e a \(U(\mathfrak{b})\)-module \(\rightsquigarrow M(\lambda) = U(\mathfrak{g}) \otimes_{U(\mathfrak{b})} F_\lambda \in \mathcal{O}^{\mathfrak{b},1} \subset \mathcal{O}^{\mathfrak{b},i}\). This is the so-called universal highest weight module. There are analogues of highest weight modules for others. Note that this is an abelian category. Quotients of Verma modules is allowed.
        
        For example, \(\mathfrak{g}=\mathfrak{sl}_2, \lambda_n \left( h = \begin{bmatrix}
            1 & 0 \\
            0 & -1 \\
        \end{bmatrix} \right) = n \in \mathbb{Z}_{\geq 0}  \rightsquigarrow 0 \to M(\lambda_{-n-2}) \hookrightarrow M(\lambda_n) \twoheadrightarrow \underbrace{V_n}_{\text{irrep of } \dim  n+1} \rightarrow 0\). 

        Remark: Category \(\mathcal{O} = \mathcal{O}^{\mathfrak{b},1}\) is \underline{not} closed under extensions in \(U\)-mod. 
        
        
        e.g. \(\mathfrak{g}= \mathfrak{sl}_2, \mathfrak{t} \to \mathfrak{gl}(\mathbb{F}^{\oplus 2}), \begin{bmatrix}
            1 & 0 \\
            0 & -1 \\
        \end{bmatrix} \mapsto \begin{bmatrix}
            0 & 1 \\
            0 & 0 \\
        \end{bmatrix}\)
        
        \[
            0 \to M(\lambda_0) \to [\mathcal{O}^{\mathfrak{b},2}\in, \mathcal{O}^{\mathfrak{b},1} \notin] U(\mathfrak{g}) \otimes_{U(\mathfrak{b})} F^{\oplus 2} \to U(\mathfrak{g}) \otimes_{U(\mathfrak{b})} F_0 = M(\lambda_0) \to 0
        \]
    \end{enumerate} 

    \subsection{Motivation} 
    
    Let \(F / \mathbb{Q}_p\) be a finite extension, \(G = \underline{G}(F)\). e.g. \(G = \operatorname{GL}_n(\mathbb{Q}_p)\).

    We would like to have a commutative diagram:

    \[
        \begin{tikzcd}
            D^{\mathfrak{b}}_{\mathcal{C}_G}(\mathcal{M}_G) \ar[r,"\mathbb{D}_G"] & D^{\mathfrak{b}}_{\mathcal{C}_G}(\mathcal{M}_G) \\
            D^{\mathfrak{b}}(\mathcal{O}^{\mathfrak{p},\infty})_{\text{alg}} \ar[u,"\mathcal{F}_P^G"] \ar[r,"\mathbb{D}_{\mathfrak{g}}"] & D^{\mathfrak{b}}(\mathcal{O}_{\text{alg}}^{\mathfrak{p},\infty}) \ar[u,"\mathcal{F}_P^G"]
        \end{tikzcd}
    \]

    Here, \(\mathcal{M}_G = D(G)\)-mod, where \(D(G)\) is the locally analytic distribution algebra.
    
    We have \(\mathbb{Q}_p[G] \overset{\text{dense}}{\hookrightarrow} D(G), U(\mathfrak{g}) \hookrightarrow D(G)\).
    
    \(\mathcal{M}_G \supset \mathcal{C}_G =\) coadmissible modules (finiteness condition).

    \(\underbrace{D^{\mathfrak{b}}_{\mathcal{C}_G}(\mathcal{M}_G)}_{\text{triangulated}} = \left\{ M^\bullet \in D^{\mathfrak{b}}(\mathcal{M}_G) \mid H^\bullet(M^\bullet) \text{ is coadmissible} \right\} \).

    \begin{theorem}
        [Schneider-Teilelbamn 2005] \(\mathbb{D}_G(M^\bullet) = \operatorname{RHom}_{D(G)} (M;D_c(G))\).
        
        \(D_c(G) =\) dual space of locally analytic functions with compact support. Dualizing objects.

        \(\mathbb{D}_G \circ \mathbb{D}_G = \operatorname{id}\). It's a right module \( \rightsquigarrow \) left module using \(g \mapsto g ^{-1}\).
    \end{theorem}

    Functors \(\mathcal{F}_P^G : \mathcal{O}_{\text{alg}}^{\mathfrak{p},\infty} \to \mathcal{C}_G \subset \mathcal{M}_G\), `globalization functors' `\(\mathfrak{g}\)-reps \( \rightsquigarrow G\)-reps'.
    
    \begin{theorem}
        [Orlib-S (2015), Agarwal-S (2021)] \(\mathcal{F}_P^G\) is exact.

        For example, if we take a Verma module \(M(\lambda)\) where \(\lambda = d_{\chi}\) [a derivative of an algebraic character],

        \[
            \mathcal{F}_B^G (M(\lambda)) = \left( \operatorname{Ind}_B^G (\chi ^{-1})^{\text{loc. an.}}\right)^\prime
        \]

        The \(\prime\) here denotes a topological dual space.

        If \(M \in \mathcal{O}^{\mathfrak{p},1}\) is simple and \(M\notin \mathcal{O}^{\mathfrak{g}, 1}, \mathfrak{g} \supset \mathfrak{p}\) then \(\mathcal{F}_P^G(M)\) is a topologically simple module. 
    \end{theorem}

    \subsection{First approach to find the duality \(\mathbb{D}_{\mathfrak{g}}\)}

    \begin{theorem}
        \(U(\mathfrak{g})\) is a noetherian ring of finite local dimnesion equal to \(\dim((\mathfrak{g}))\).
        
        \(U = U(\mathfrak{g})\) is therefore a dualizing object itself. i.e. the functor \(\operatorname{RHom}_U(-,U)\) is an involutive anti-isomorphism of \(D^{\mathfrak{b}}(U\text{-mod})\).  
    \end{theorem}

    Fact: \(\forall M \in \mathcal{O}^{\mathfrak{p},\infty}: \forall q \geq 0:\) \(\operatorname{Ext}^q_U(M,U) \in \mathcal{O}^{\mathfrak{p}, i}\)

    Question: If \(M\in \mathcal{O}^{\mathfrak{p, \infty}}:\) is \(\operatorname{RHom}_U(M,U)\) naturally quasi-isomorphic to a complex of modules in \(\mathcal{O}^{\mathfrak{p}, \infty}\)?
    
    Answer: Yes, if \(\mathfrak{p} = \mathfrak{b}\).

    \begin{theorem}
        [Coulembier - Mazorchuk] The category \(\mathcal{O}^{\mathfrak{b},\infty}\) is \underline{extension-full} in \(U\)-mod.
        
        Extension-full means, for all objects \(M,N \in \mathcal{O}^{\mathfrak{b},\infty}\) and \(\forall q\): the Yoneda Ext group \(\operatorname{Y-Ext}^q_{\mathcal{O}^{b,\infty}} (M,N) \xrightarrow{\cong} \operatorname{Ext}_{U\text{-mod}}^q (M,U)\).
        
        We need Yoneda Ext Group since the category doesn't have enough projective and injective objects.
        
    \end{theorem}

    \begin{corollary}
        The natural functor \(D^{\mathfrak{b}}(\mathcal{O}^{\mathfrak{b},\infty}) \to D_{\mathcal{O}^{\mathfrak{b},\infty}}^{\mathfrak{b}}(U \text{-mod})\) is an equivalence of categories. The subscript \(\mathcal{O}^{\mathfrak{b},\infty}\) means the cohomology is in \(\mathcal{O}^{\mathfrak{b},\infty}\).
        
        \(\mathbb{D}\) acts on \(D^{\mathfrak{b}}(\mathcal{O}^{\mathfrak{b},\infty})\) by transport of structure.
        
        \(\mathbb{D}\) acts on \(D^{\mathfrak{b}}_{\mathcal{O}^{\mathfrak{b},\infty}}\).  
    \end{corollary}

    \subsection{Second approach}

    First approach doesn't generalize to general parabolic subalgebras \(\mathfrak{p}\). Take \(\mathfrak{p} = \mathfrak{g} \) semisimple \(= \mathcal{O}^{\mathfrak{g} , \infty} = \mathcal{O}^{\mathfrak{g},1}\) which is the category of all f.g. \(g\)-modules.
    
    
    \(\implies \operatorname{Ext}^q_{\mathcal{O}^{\mathfrak{g}}} (M,N) = 0\) for all \(q > 0\).
    
    
    Even for \(\mathfrak{sl}_2: H^3(\mathfrak{g},M) \neq 0\).

    Note \(H^3(\mathfrak{g},M) = \operatorname{Ext}_{\mathfrak{g}}^3 (\mathbf{1}, M)\).

    Solution in general:

    \(M \mapsto \operatorname{RHom}_U(M,U)\)
    
    \(M \mapsto \operatorname{Hom}_U(M,U) = 0\) 

    \(M \mapsto \operatorname{Ext}^e_U(M,U) = E^{\mathfrak{p}}(M)\) where \(e = \dim \mathfrak{p} \).
    
    \(\operatorname{Ext}_U^{\operatorname{dim}(\mathfrak{p})} \left( U \otimes_{U(\mathfrak{p})} \underbrace{W}_{\text{f.d.}} , U \right) \cong \left( W^{\ast} \otimes \land^{\text{top}} \mathfrak{p}^{\ast}  \right) \otimes_{U(\mathfrak{p})} U\) 

    \begin{theorem}
        \(\operatorname{RHom}_U(M,U)\) [here \(M\in \mathcal{O}^{\mathfrak{p},\infty}\)] equals \(\operatorname{RE}^{\mathfrak{p}}(M)[-e]^{\mathcal{O}^{\mathfrak{b},\infty}}\)
        
        Superscript denotes the cohomology is in \(\mathcal{O}^{\mathfrak{b},\infty}\).
    \end{theorem}

\end{document}